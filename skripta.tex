\documentclass[12pt,a4paper]{article}
\usepackage[T1]{fontenc}
\usepackage[utf8]{inputenc}
\usepackage[czech]{babel}
\usepackage{amsfonts}
\usepackage{amssymb}
\usepackage{amsmath}
\usepackage{amsthm}

\addtolength{\textwidth}{2in}
\addtolength{\hoffset}{-1in}
\addtolength{\textheight}{1in}
\addtolength{\voffset}{-1in}

\newcounter{vety}
\newtheorem*{definice}{Definice}
\newtheorem{vetal}[vety]{Věta L}
\newtheorem{vetat}[vety]{Věta T}
\newtheorem{vetabd}[vety]{Věta BD}
\newtheorem*{dusledek}{Důsledek}
\newtheorem*{opakovani}{Opakování}
\newtheorem*{pozorovani}{Pozorování}
\newtheorem*{poznamka}{Poznámka}
\newtheorem*{priklad}{Příklad}


\newcommand{\con}{\rightrightarrows}
\newcommand{\conloc}{\overset{loc}{\rightrightarrows}}

\title{Matematická analýza III}
\author{Tomáš Krejčí <tomas789@gmail.com>}

\begin{document}

\maketitle

\setcounter{section}{9}
\section{Konvergence posloupností a řad funkcí}

\subsection{Bodová a stejnoměrná konvergence posloupnosti funkcí}

\begin{definice}
Nechť $J \subset \mathbb{R}$ je interval a nechť máme $f : J \rightarrow \mathbb{R}$ a $f_n : J \rightarrow \mathbb{R}$ pro $n \in \mathbb{N}$. Řekneme, že posloupnost funkcí $\{f_n\}$:

\begin{enumerate}
\item \emph{konverguje bodově} k $f$ na $J$, pokud pro každé $x \in J$ platí $\lim_{n \rightarrow \infty} f_n(x) = f(x)$, neboli
$$\forall x \in J \forall \epsilon > 0 \exists n_0 \in \mathbb{N} \forall n \geq n_0 : |f_n(x) - f(x)| < \epsilon$$
Značíme $f_n \rightarrow f$ na $J$.
\item \emph{konverguje stejnoměrně} k $f$ na $J$, pokud
$$\forall \epsilon > 0 \exists n_0 \in \mathbb{N} \forall n \geq n_0 \forall x \in J : | f_n(x) - f(x) | < \epsilon$$
Značíme $f_n \con f$.
\item \emph{konverguje lokálně stejnoměrně}, pokud pro každý omezený uzavřený interval $[a, b] \subset J$ platí: $f_n \rightrightarrows f$ na $[a, b]$. Značíme $f_n \conloc f$
\end{enumerate}
\end{definice}

\begin{vetal}[kritérium stejnoměrné konvergence]
Nechť $f_n, f:J \rightarrow \mathbb{R}$ pak
$$f_n \con f na J \Leftrightarrow \lim_{n \rightarrow \inf} \sup \{ |f_n(x) = f(x)|; x \in J \} = 0$$
\end{vetal}

\begin{vetat}[Bolzano-Cauchyho podmínka pro stejnoměrnou konvergenci]
Nechť $f_n,f : J \rightarrow \mathbb{R}$. Pak
$$f_n \con f na J \Leftrightarrow \forall \epsilon > 0 \exists n_0 \forall m,n \geq n_0 \forall x \in J : | f_n(x) - f_m(x)| < \epsilon$$
\end{vetat}

\begin{vetat}[Moore-Osgood]
Nechť $x_0$ je krajní bod intervalu J (může být i $\pm \infty$). Nechť $f, f_n : J \rightarrow \mathbb{R}$ splňují
\begin{enumerate}
\item $f_n \con f$ na $J$,
\item existuje $\lim_{x \rightarrow x_0} f_n(x) = a_n \in \mathbb{R}$ pro všechna $n \in \mathbb{N}$
\end{enumerate}
Pak existují $\lim_{n \rightarrow \infty} a_n$ a $\lim_{x \rightarrow x_0} f(x)$ a jsou si rovny, neboli:
$$\lim_{n \rightarrow \infty} \lim_{x \rightarrow x_0} f_n(x) = \lim_{x \rightarrow x_0} \lim_{n \rightarrow \infty} f_n(x)$$
\end{vetat}

\begin{dusledek}
Nechť $f_n \con f$ na $I$ a nechť $f_n$ jsou na $I$ spojité. Pak $f$ je spojitá na $I$.
\end{dusledek}

\begin{vetal}[o záměně limity a integrálu]
Nechť funkce $f_n \con f$ na $[a,b]$ a nechť $f_n \in \mathbb{R} ([a,b])$. Pak $f \in \mathbb{R}([a,b])$ a 
$$(R) \int_a^b f(x) dx = \lim_{n \rightarrow \infty} (R) \int_a^b f_n(x) dx$$
\end{vetal}

\begin{vetat}[o záměně limity a derivace]
Nechť funkce $f_n$, $n \in \mathbb{N}$, mají vlastní derivaci na intervalu $(a,b)$ a nechť:
\begin{enumerate}
\item existuje $x_0 \in (a,b)$ tak, že $\{f_n(x_0)\}_{n=0}^{\infty}$ konverguje,
\item pro derivace $f_n'$ platí $f_n' \conloc$ na $(a,b)$
\end{enumerate}
Potom existuje funkce $f$ tak, že $f_n \conloc f$ na $(a,b)$, $f$ má vlastní derivaci a platí $f_n' \conloc f'$ na $(a,b)$.
\end{vetat}

\subsection{Stejnoměrná konvergence řady funkcí}
\begin{definice}
Řekneme, že řada funkcí $\sum_{k=1}^{\infty} u_k (x)$ konverguje \emph{stejnoměrně} (popřípadě \emph{lokálně stejnoměrne}) na intervalu $J$, pokud posloupnost částečných součtů $s_n(x) = \sum_{k=1}^{n} u_k (x)$ konverguje stejnoměrně (popřípadě lokálně stejnoměrně) na $J$.
\end{definice}

\begin{vetal}[nutná podmínka stejnoměrné konvergence řady]
Nechť $\sum_{n=1}^{\infty} u_k (x)$ je řada funkcí definovaná na intervalu $J$. Pokud $\sum_{k=1}^{\infty} u_n \con$ na $J$, pak posloupnost funkcí $u_n (x) \con 0$ na $J$.
\end{vetal}

\begin{vetal}[Weirstrassovo kritérium]
Nechť $\sum_{k=1}^{\infty} u_n (x)$ je řada funkcí definovaná na intervalu $J$. Pokud pro $a_n := \sup \{ | u_n (x) |; x \in J \}$ platí, že číselná řada $\sum_{n=1}^{\infty} a_n$ konverguje, pak $\sum_{n=1}^{\infty} u_n \con$ na $J$.
\end{vetal}

\begin{vetal}[o spojitosti a derivování řad funkcí]
Nechť $\sum_{n=1}^{\infty} u_n (x)$ je řada funkcí definovaná na intervalu $(a,b)$.
\begin{enumerate}
\item Nechť $u_n$ jsou spojité na $(a,b)$ a nechť $\sum_{n=1}^{\infty} u_n (x) \conloc$ na $(a,b)$. Pak $F (x) = \sum_{n=1}^{\infty} u_n (x)$ je spojitá na $(a,b)$.
\item Nechť funkce $u_n$, $n \in \mathbb{N}$ mají vlastní derivace na intervalu $(a,b)$ a nechť
	\begin{enumerate}
	\item existuje $x_0 \in (a,b)$ tak, že $\sum_{n=1}^{\infty} u_n (x_0)$ konverguje,
	\item pro derivace $u_n'$ platí $\sum_{n=1}^{\infty} u_n' \conloc$ na $(a,b)$
	\end{enumerate}
\end{enumerate}
Potom je funkce $F(x) = \sum_{n=1}^{\infty} u_n (x)$ dobře definovaná diferencovatelná a navíc $\sum_{n=1}^{\infty} u_n (x) \conloc F(x)$ a $\sum_{n=1}^{\infty} u'_n (x) \conloc F'(x)$ na $(a,b)$.
\end{vetal}

Vraťme se ke konvergenci obyčejných řad. Následující kritérium bude užitečné v kapitole Fourierovy řady. Existuje i varianta tohoto tvrzení pro stejnoměrnou konvergenci, tu však nebudeme potřebovat.

\begin{vetat}[Abel-Dirichletovo kriterium, bez důkazu]
Nechť $\{a_n\}_{n \in \mathbb{N}}$ je posloupnost reálných čísel a $\{b_n\}_{n=1}^{\infty}$ je nerostoucí posloupnost nezáporných čísel. Jestliže je některá z následujících podmínek splněna, pak je $\sum_{n=1}^{\infty} a_n b_n$ konvergentní.
\begin{enumerate}
\item $\sum_{n=1}^{\infty} a_n$ je konvergentní,
\item $\lim_{n \rightarrow \infty} b_n = 0$ a $\sum_{n=1}^{\infty} a_n$ má omezené součty, tedy
$$\exists K > 0 \quad \forall m \in \mathbb{N} \quad : | s_m | = \left| \sum_{i=1}^{m} a_i  \right| < K$$
\end{enumerate}
\end{vetat}

\pagebreak
\setcounter{vety}{0}
\section{Mocninné řady}

\begin{definice}
Nehcť $x_0 \in \mathbb{R}$ a $a_n \in \mathbb{R}$ pro $n \in \mathbb{N}_0$. Řadu funkcí $\sum_{n=0}^{\infty} a_n (x-x_0)^n$ nazýváme \emph{mocninnou řadou} s koeficienty $a_n$ o středu $x_0$.
\end{definice}

\begin{definice}
\emph{Poloměrem konvergence} mocninné řady $\sum_{n=0}^{\infty} a_n (x-x_0)^n$ nazveme $$R = \sup \left\{ r \in [ 0,\infty ) : \sum_{n=0}^{\infty}  a_n ( x - x_0 )^n \textrm{ konverguje } \forall x \in [ x_0 - r; x_0 + r ] \right\}$$
\end{definice}

\begin{vetal}[o poloměru konvergence mocninné řady]
Nechť $\sum_{n=0}^{\infty} a_n (x-x_0)^n$ je mocninná řada a $R \in [ 0, \infty ]$ její poloměr konvergence. Pak řada konverguje obsolutně pro všechna $x$ taková, že $| x - x_0| < R$ a diverguje pro všechna $x$ taková, že $| x - x_0 | > R$.
\end{vetal}

\begin{vetal}[výpočet poloměru konvergence]
Nechť $\sum_{n=0}^{\infty} a_n (x-x_0)^n$ je mocninná řada a $R \in [ 0, \infty ]$ její poloměr konvergence. Pak platí
$$R = \frac{1}{ \lim \sup_{n \rightarrow \infty} \sqrt[n]{| a_n |} }$$
Pokud existuje $\lim_{n \rightarrow \infty} \frac{|a_n|}{|a_{n+1}|}$, pak $R = \lim_{n \rightarrow \infty} \frac{|a_n|}{|a_{n+1}|}$.
\end{vetal}

\begin{vetal}
Nechť $\sum_{n=0}^{\infty} a_n (x-x_0)^n$ je mocninná řada s poloměrem konvergence $R > 0$.  Pak řada konverguje lokálně stejnoměrně na $(x_0 - R, x_0 + R)$ (je-li $R=\infty$, pak na celém $\mathbb{R}$).
\end{vetal}

\begin{vetal}[o derivaci mocninné řady]
Nechť $\sum_{n=0}^{\infty} a_n (x-x_0)^n$ je mocninná řada s poloměrem konvergence $R > 0$. Pak $\sum_{n=1}^{\infty} n a_n (x-x_0)^{n-1}$ je také mocninná řada se stejným středem a poloměrem konvergence. Navíc pro $x \in ( x_0 - R, x_0 + R )$ ($\mathbb{R}$ pro $R = \infty$) platí
$$ \left( \sum_{n=0}^{\infty} a_n (x-x_0)^n \right)' = \sum_{n=+}^{\infty} n a_n (x-x_0)^{n-1}$$
\end{vetal}

\begin{vetal}[o integrování mocninné řady]
Nehcť $\sum_{n=0}^{\infty} a_n (x-x_0)^n$ je mocninná řada s poloměrem konvergence $R > 0$. Pak $\sum_{n=0}^{\infty} \frac{a_n}{n+1} \left(x-x_0 \right)^{n+1}$ je také mocninná řada se stejným poloměrem konvergence. Navíc platí
$$\int \sum_{n=0}^{\infty} a_n (x-x_0)^n dx = \sum_{n=1}^{\infty} \frac{a_n}{n+1} \left( x-x_0 \right)^{n+1} + C  \quad \textrm{na } \left( x_0 - R, x_0 + R \right)$$
\end{vetal}

\begin{vetat}[Abelova]
Nechť $\sum_{n=0}^{\infty} a_n (x-x_0)^n$ je mocninná řada s poloměrem konvergence $R > 0$. Nechť navíc $\sum_{n=0}^{\infty} a_n R^n$ konverguje. Potom řada $\sum_{n=0}^{\infty} a_n (x-x_0)^n$ konverguje stejnoměrně na $[ x_0, x_0 + R ]$ a platí
$$\sum_{n=0}^{\infty} a_n R^n = \lim_{r \rightarrow R_-} \sum_{n=0}^{\infty} a_n r^n$$
\end{vetat}

\begin{proof}
Předpokládejme bez újmy na obecnosti, že $x_0 = 0$. Označme $t_N = \sum_{n=N+1}^{\infty} a_n R^n$. Víme, že $\sum a_n R^n$ konverguje, tedy
$$\forall \varepsilon>0 \quad \exists n_0 \quad \forall n \geq n_0 \quad : \quad |t_N| < \varepsilon$$
\begin{eqnarray}
a_n & = & a_n R^n \left( \frac{x}{R} \right)^n \nonumber\\
& = & - t_N \left( \left( \frac{x}{R} \right)^n - \left( \frac{x}{R} \right)^{n+1} \right) + t_{n-1} \left( \frac{x}{R} \right)^n - t_n \left( \frac{x}{R} \right)^{n+1} \nonumber
\end{eqnarray}
Sečteme od $N$ do $N+k$
$$\sum_{n=N}^{N+k} a_n x^n = \left[ \sum_{n=N}^{N+k} -t_n \left( \left( \frac{x}{R} \right)^n - \left( \frac{x}{R} \right)^{n+1} \right) \right] + t_{N-1} \left( \frac{x}{R} \right)^n - t_{N+k} \left( \frac{x}{R} \right)^{n+k+1}$$
Protože $x \in [0, R]$, tak $\left( \frac{x}{R} \right)^n \in [0,1]$. Dále platí $\left( \frac{x}{R} \right)^{n} - \left( \frac{x}{R} \right)^{n+1} \geq 0$.
\begin{eqnarray}
\left|\sum_{n=N}^{N+k} a_n x^n \right| & \leq & \sum_{n=N}^{N+k} |t_n| \left( \left( \frac{x}{R} \right)^{n} - \left( \frac{x}{R} \right)^{n+1} \right) + |t_{N-1}| + |t_{N+k}| \nonumber\\
& \leq & \epsilon \sum_{n=N}^{N+k} |t_n| \left( \left( \frac{x}{R} \right)^{n} - \left( \frac{x}{R} \right)^{n+1} \right) + 2 \varepsilon \nonumber\\
& = & \varepsilon \left( \left( \frac{x}{R} \right)^{N} - \left( \frac{x}{R} \right)^{N+k+1} \right) + 2 \varepsilon \nonumber\\
& \leq & 3 \varepsilon \nonumber
\end{eqnarray}
Z \emph{BC} podmínky pro stejnoměrnou konvergenci řady dostaneme $\sum_{n=0}^{\infty} a_n x^n \con$ na $[0, R]$
Z MO věty dostaneme
$$\lim_{n \rightarrow \infty} \lim_{r \rightarrow R_-} \sum_{n=0}^{N} a_n R^n = \lim_{r \rightarrow R_-} \lim_{n \rightarrow \infty} \sum_{n=0}^{N} a_n R^n$$
$$\lim_{n \rightarrow \infty} \lim_{r \rightarrow R_-} \sum_{n=0}^{N} a_n R^n = \lim_{n \rightarrow \infty} \sum_{n=0}^{N} a_n R^n = \sum_{n=0}^{\infty} a_n R^n$$
$$\lim_{r \rightarrow R_-} \lim_{n \rightarrow \infty} \sum_{n=0}^{N} a_n R^n = \lim_{r \rightarrow R_-} \sum_{n=0}^{\infty} a_n r^n$$
\end{proof}

\begin{priklad}
Sečtěte $\sum_{n=1}^{\infty} \frac{(-1)^{n-1}}{3n-2}$
\end{priklad}

\begin{proof}[Řešení]
Nechť $f(x) = \sum_{n=1}^{\infty} \frac{(-1)^{n-1}}{3n-2} x^{3n-2}$. To je mocninná řada poloměrem konvergence $R=1$. Podle Laibnitze $f(1) = \sum_{n=1}^{\infty} \frac{(-1)^{n-1}}{3n-2}$ konverguje. 

Tedy podle Abelovy věty $\sum_{n=1}^{\infty} \frac{(-1)^{n-1}}{3n-2} = \lim_{x \rightarrow 1_-} f(x)$

Dle věty o derivaci mocninné řady máme $\forall x \in (-1, 1)$
$$f'(x) = \sum_{n=1}^{\infty} \frac{(-1)^{n-1}}{3n-2} (3n-2) x^{3n-3} = \sum_{n=1}^{\infty} \left( -x^3 \right)^{n-1} = \frac{1}{1+x^3}$$
$$f(x) = \int \frac{1}{1+x^3} dx = \ldots = \frac{1}{3} \ln(x+1) - \frac{1}{6} \ln(x^2-x+1) + \frac{1}{\sqrt{3}} \arctan \left( \frac{2x-1}{\sqrt{3}} \right) + C$$
$$0 = f(0) = \frac{1}{3} 0 - \frac{1}{6} 0 + \frac{1}{\sqrt{3}} \arctan \left( - \frac{1}{\sqrt{3}} \right) + C \Rightarrow C = \frac{1}{\sqrt{3}} \arctan \left( \frac{1}{\sqrt{3}} \right)$$
\begin{eqnarray}
\sum_{n=1}^{\infty} \frac{(-1)^{n-1}}{3n-2} & = & \lim_{x \rightarrow 1_-} \left( \frac{1}{3} \ln(x+1) - \frac{1}{6} \ln(x^2-x+1) + \frac{1}{\sqrt{3}} \arctan \left( \frac{2x-1}{\sqrt{3}} \right) \frac{1}{\sqrt{3}} \arctan \left( \frac{1}{\sqrt{3}} \right) \right) \nonumber\\
& = & \frac{1}{3} \ln(2) + \frac{2}{\sqrt{3}} \arctan \left( \frac{1}{\sqrt{3}} \right) \nonumber
\end{eqnarray}
\end{proof}

\pagebreak
\setcounter{vety}{0}
\section{Fourierovy řady}

\begin{definice}
Nechť $a_k \in \mathbb{R}$ pro $k \in \mathbb{N}_0$ a $b_k \in \mathbb{R}$ pro $k \in \mathbb{N}$. Řadu $\frac{a_0}{2} + \sum_{k=1}^{\infty} \left( a_k \cos (kx) + b_k \sin (kx) \right)$ pro $x \in \mathbb{R}$ nazveme \emph{trigoniometrickou řadou}. Pro dané $n$ je $\frac{a_0}{2} + \sum_{k=1}^{n} \left( a_k \cos (kx) + b_k \sin (kx) \right)$   \emph{trigoniometrický polynom stupně $n$}. $\mathcal{P}_{2 \pi}$ značí množinu všech $2 \pi$-periodických funkcí majících Reimannův integrál na $[0, 2 \pi]$
\end{definice}

Cílem je danou $f \in \mathcal{P}_{2 \pi}$ rozvinout do trigoniometrické řady a:
\begin{enumerate}
\item spočítat $a_k$, $b_k$
\item zjistit, zda-li je řada rovna původní funkci
\end{enumerate}

\begin{vetabd}
Nechť $\{ a_n \}_{n \in \mathbb{N}}$ je posloupnost reálných čísel a $\{ b_n \}_{n=1}^{\infty}$ je nerostoucí posloupnost reálných čísel. Jestliže buď
\begin{enumerate}
\item[(A)] $\sum_{n=1}^{\infty} a_n$ je konvergentní, nebo
\item[(D)] $\lim_{n \rightarrow \infty} b_n = 0$ a $\sum_{n=1}^{\infty} a_n$ má omezené částečné součty
\end{enumerate}
Pak $\sum_{n=1}^{\infty} a_n b_n$ konverguje.
\end{vetabd}

\begin{priklad}
Vyšetřete konvergenci řady
$$\sum_{n=1}^\infty \frac{\sin (nx)}{n}$$
\end{priklad}

\begin{proof}[Řešení]
Pokud $x= \pi k$, $k \in \mathbb{Z} \Leftrightarrow \sum 0 \leftarrow$ konverguje.

Dále předpokládejme, že $x \neq \pi k$. Označme $a_n = \sin (nx)$, $b_n = \frac{1}{n}$. $b_n$ je monotonní nerostoucí a $\lim_{n \rightarrow \infty} b_n = 0$. Nechť $m \in \mathbb{N}$.

$$\left | \sum_{n=a}^{m} \sin (nx) \right | = \left | Im \left( \sum_{n=0}^{m} e^{inx} \right ) \right | = \left | Im \left( \frac{1- \left( e^{ix} \right)^{n+1}  }{1-e^{ix}} \right) \right | \leq \frac{3}{ \left | 1-e^{ix} \right | }$$

Dle Dirichletova kriteria tato suma konverguje.

\end{proof}

\begin{vetal}[o ortogonalitě trigoniometrických funkcí]
Nechť $m, n \in \mathbb{N}$, pak
\begin{eqnarray}
\int_{0}^{2 \pi} \sin (nx) \cos (mx) dx & = & 0 \nonumber\\
\int_{0}^{2 \pi} \sin (nx) \sin (mx) dx & = & \pi \textrm{ (pro $n=m$)}, 0 \textrm{ (pro $n \neq m$) } \nonumber\\
\int_{0}^{2 \pi} \cos (nx) \cos (mx) dx & = & \pi \textrm{ (pro $n \neq m$)}, 0 \textrm{ (pro $n=m$)} \nonumber
\end{eqnarray}
\end{vetal}

\emph{Poznámka} proč se věta jmenuje o ortogonalitě trigoniometrických funkcí? Vraťme se zpět k lineární algebře. Skalární součin vektorů $x$ a $y$ jsme definovali jako $\left\langle x,y \right\rangle = \sum_{i=0}^{n} x_i y_i$. Zcela ekvivalentně byl zaveden skalární součit funkcí $f$ a $g$ jako $\left\langle f, g \right\rangle = \int f(x) g(x) dx$. O vektorech řekneme, že jsou na sebe kolmé (ortogonální), pokud je jejich skalární součin roven nule. Nejinak je tomu i u skalárního součinu funkcí. 

\emph{Poznámka 2} Skalární součin funkcí se nazývá Hilbertovy prostory

\begin{proof}

\begin{eqnarray}
\sin \alpha \sin \beta & = & \frac{1}{2} \left( \cos (\alpha - \beta) - \cos (\alpha + \beta) \right) \nonumber\\
\cos \alpha \cos \beta & = & \frac{1}{2} \left( \cos (\alpha - \beta) + \cos (\alpha + \beta) \right) \nonumber\\
\sin \alpha \cos \beta & = & \frac{1}{2} \left( \cos (\alpha - \beta) + \sin (\alpha + \beta) \right) \nonumber
\end{eqnarray}

$$\int_{0}^{2 \pi} \sin (nx) \cos(mx) = \int_{0}^{2 \pi} \left[ \frac{1}{2} \cos \left( (n-m)x \right) - \frac{1}{2} \left( (n+m)x \right) \right] = $$
Pro $n \neq m$
$$= \left[ \frac{1}{2} \frac{\sin \left( (n-m)x \right)}{n-m} \right]_{0}^{2 \pi} - \left[ \frac{1}{2} \frac{\sin \left( (n+m)x \right)}{n+m} \right]_{0}^{2 \pi} = 0$$
Pro $n=m$
$$=\int_{0}^{2 \pi} \frac{1}{2} \cos 0 + (2. clen stejne) = \pi$$

Zbylé rovnosti analogicky.
\end{proof}\begin{opakovani}[Vlastnosti Reimanovsky integrovantelných funkcí] \quad
\begin{enumerate}
\item $f \in R((a,b)) \Leftrightarrow \forall \varepsilon > 0 \textrm{ dělení } (a,b)$ : $S(f,D) -s(f,D) < \varepsilon$
\item $f \in R((a,b))$ a $f \in R((b,c)) \Leftrightarrow f \in R((a,c))$ pro $a<b<c$
\item $f$ je spojitá na $[a,b] \Rightarrow f \in R((a,b))$
\item $f$ je spojitá na $(a,b)$ a omezená na $[a,b] \Rightarrow f \in R((a,b))$
\item $f, g \in R((a,b)) \Rightarrow f \pm g, f * g \in R((a,b))$
\end{enumerate}
\end{opakovani}

\begin{vetal}[Fourierovy vzorce]
Nechť $f \in \mathcal{P}_{2 \pi}$ a nechť $f(x) = \frac{a_0}{2} + \sum_{k=1}^{\infty} a_k \cos(kx) + b_k \sin(kx)$, nechť navíc řada napravo konverguje stejnoměrně. Pak
\begin{eqnarray}
a_k & = & \frac{1}{\pi} \int_0^{2 \pi} f(x) \cos(kx) dx, \quad k \in \{0, 1, 2, \ldots\} \nonumber\\
b_k & = & \frac{1}{\pi} \int_0^{2 \pi} f(x) \sin(kx) dx, \quad k \in \{1, 2, \ldots\} \nonumber
\end{eqnarray}
\end{vetal}

\begin{proof}
Idea důkazu je, že identitu pro $f(x)$ přenásobíme $\cos(kx)$ resp. $\sin(kx)$  a přeintegrujeme přes $[0, 2 \pi]$ a díky větě "Věta L 2" (! odpovídá značení na přednášce, nikoliv v tomto skriptu) mnoho členů vypadne.

\begin{opakovani}
$f_n \con f$ na $(a,b) \Rightarrow \int_a^b f_n \rightarrow \int_a^b f$
\end{opakovani}

\begin{pozorovani}
$\sum_{k=1}^{\infty} \left( a_k \cos(kx) \sin(lx) + b_k \sin(kx) \sin(lx) \right)$ konverguje stejnoměrně.
\end{pozorovani}

$$\int_{0}^{2 \pi} f(x) \sin(lx) dx = \int_{0}^{2 \pi} \left[ \frac{a_0}{2} \sin(lx) + \sum \left( a_k \cos(kx) \sin(lx) + b_k \sin(kx) \sin(lx) \right) \right] dx = $$
$$b_k \int_{0}^{2 \pi} \sin^2(lx) dx = b_k \pi \Rightarrow b_k = \frac{1}{\pi} \int_0^{2 \pi} f(x) \sin(lx) dx$$

podobně přenásobím funkcí $\sin(lx)$ dostanu vzorec pro $a_k$ pro $k \in \mathbb{N}$.

Přenásobím funkcí $\cos(0x) = 1$:

$$\int_{0}^{2 \pi} 1 * f(x) dx = \int_{0}^{2 \pi} \frac{a_0}{2}+0+0+\ldots+0 \Rightarrow a_0 = \frac{1}{\pi} \int_{0}^{2 \pi} f(x) dx = \frac{1}{\pi} \int_0^{2 \pi} f(x) \cos(0x) dx$$

\end{proof}

\begin{definice}
Nehchť $f \in \mathcal{P}_{2 \pi}$. Pak definujeme čísla

\begin{eqnarray}
a_k & = & \frac{1}{\pi} \int_{0}^{2 \pi} f(x) \cos(kx) dx, \quad k=0,1,\ldots \nonumber\\
b_k & = & \frac{1}{\pi} \int_{0}^{2 \pi} f(x) \sin(kx) dx, \quad k=1,2,\ldots \nonumber
\end{eqnarray}

a nazveme je \emph{Fourierovými koeficienty} funkce $f$ a 

$$a F_f = \frac{a_0}{2} + \sum_{k=1}^\infty a_k \cos(kx) + b_k \sin(kx)$$

nazveme Fourierovou řadou funkce $f$.
\end{definice}

\begin{poznamka}
\begin{itemize} \quad
\item díky $2 \pi$-periodicitě lze funkci integrovat přes libovolný interval délky $2 \pi$ (velmi často $\int_{-\pi}^{\pi}$)
\item Fourierovy řady lze zavést i pro funkce s periodou $l$, pak mají vzorce tvar
\begin{eqnarray}
F_f & = & \frac{a_0}{2} + \sum_{k=1}^{\infty} a_k \cos \left( \frac{2 k \pi}{l} x \right) + b_k \sin \left( \frac{2 k \pi}{l} x \right) \nonumber\\
a_k & = & \frac{2}{l} \int_0^{l} f(x) \cos \left( \frac{2 k \pi}{l} x \right) dx \nonumber\\
b_k & = & \frac{2}{l} \int_0^l f(x) \sin \left( \frac{2 k \pi}{l} x \right) dx \nonumber
\end{eqnarray}
\item někdy se pracuje s rozvoji vůči jinému systému než je trn trigoniometrický
\item je-li $f$ lichá, pak platí $\forall k : a_k = 0$
\item je-li $f$ sudá, pak platí $\forall k : b_k = 0$
\item opecně neplatí $F_f = f$
\end{itemize}
\end{poznamka}

\begin{priklad}
Rozviňte funkci $f(x) = x^2$ do Fourierovy řady na $(-\pi, \pi)$.
\end{priklad}

Funkce $f$ je sudá $\Rightarrow \forall k b_k = 0$.

\begin{eqnarray}
a_0 & = & \frac{1}{\pi} \int_{-\pi}^{\pi} f(x) dx = \frac{2}{\pi} \int_0^{\pi} x^2 dx = \frac{2}{\pi} \left[ \frac{x^3}{3} \right]_0^\pi = \frac{2}{3} \pi^2 \nonumber\\
a_k & = & \frac{2}{\pi} \int_{0}^{\pi} x^2 \cos (kx) dx = \frac{2}{\pi} \left( \left[x^2 \frac{\sin (kx)}{k} \right]_0^\pi - \int_0^\pi 2x \frac{\sin (kx)}{k} dx \right) \nonumber\\
& = & \frac{2}{\pi} \left( 0 - 0 - \frac{2}{k} \int_0^\pi x \sin (kx) dx \right) = \frac{4}{k^2 \pi} \left( \left[ x \frac{\cos (kx)}{k} \right]_0^\pi - \int_0^\pi 1 * \frac{\cos(kx)}{k} dx \right) \nonumber\\
& = & \frac{4}{k^2 \pi} \left( \pi \cos (k \pi) - 0 - 0 \right) = \frac{4}{k^2} \cos (k \pi) = \frac{4}{k^2} (-1)^k \nonumber
\end{eqnarray}

$$F_f(x) = \frac{1}{3} \pi^2 + \sum_{k=1}^{\infty} \frac{4}{k^2} (-1)^k \cos (kx)$$

\begin{definice}
Nechť $n \in \mathbb{N}$. Pak \emph{Dirichletovým jádrem} nazveme funkci
$$D_n(x) = \frac{1}{2}+\cos(x)+\cos(2x)+\ldots+\cos(nx)$$
\end{definice}

\begin{dusledek}
\begin{enumerate}
\item $D_n$ je spojitá funkce, sudá, $2 \pi$-periodická, $D_n(0) = n + \frac{1}{2}$
\item $\int_{-\pi}^\pi D_n(x) dx = \pi$
\item $D_n(x) = \frac{\sin \left( n + \frac{1}{2} x \right)}{2 \sin \frac{x}{2}}, \quad \forall x \in \mathbb{R} \backslash \{ 2 k \pi \}_{k \in \mathbb{Z}}$
\end{enumerate}
\end{dusledek}

\begin{proof}
\begin{enumerate}
\item č
\end{enumerate}
\end{proof}

\begin{vetal}[vlastnosti Dirichletova jádra]
Pro Dirichletovo jádro $D_n$ platí
\begin{enumerate}
\item $D_n$ je spojitá, sudá, $2 \pi$-periodická funkce
\item $\int_{-\infty}^{\infty} D_n (x) dx = \pi$
\item $D_n(x) = \frac{\sin \left( n + \frac{1}{2} \right) x}{2 \sin \left( \frac{x}{2} \right)}$, $\forall x \in \mathbb{R} \backslash \bigcup_{k \in \mathbb{Z}} \{ 2k \pi \}$
\end{enumerate}
\end{vetal}

\begin{vetal}[částečné součty Fourierovy řady]
Nechť $f \in \mathcal{P}_{2 \pi}$ a $F_f$ je Fourierova řada pro $f$. Potom pro částečné součty $s_n(x) = \frac{a_0}{2} + \sum_{k=1}^{n} (a_k \cos ( kx) + b_k \sin ( kx ))$ platí
$$s_n(x) = \frac{1}{\pi} \int_{-pi}^\pi f(x+z)D_n(y)dy = \frac{1}{\pi} \int_0^\pi \left( f(x+y) + f(x-y) \right) D_n(y) dy$$
\end{vetal}

\begin{vetat}[Riemann-Lebesqueovo lemma]
Nechť $[a,b] \subset \mathbb{R}$ a $f \in R([a,b])$. Potom
$$\lim_{t \rightarrow \infty} \int_a^b f(x) \cos(tx) dx = 0 \mathrm{a} \lim_{t \rightarrow \infty} \int_a^b f(x) \sin(tx) dx = 0$$
Speciálně pro Fourierovy koeficienty funkce $f \in \mathcal{P}_{2 \pi}$ platí $a_k \rightarrow 0$ a $b_k \rightarrow 0$.
\end{vetat}

\begin{vetat}[Diniho kriterium]
Nechť $f \int \mathcal{P}_{2 \pi}$ a $x \in \mathbb{R}$. Nechť existují vlastní limity $f(x+) = \lim_{y \rightarrow x+} f(y)$ a $f(x-) = \lim_{y \rightarrow x-} f(y)$ a nechť existují vlastní limity
$$\lim_{y \rightarrow x+} \frac{f(y)-f(x+)}{y-x} \quad \mathrm{a} \quad \lim_{y \rightarrow x-} \frac{f(y)-f(x-)}{y-x}$$
Potom Fourierova řada funkce $f$ konverguje v bodě $x$ k hodnotě $\frac{f(x+) + f(x-)}{2}$.
\end{vetat}

\begin{dusledek}
Nechť $x \in \mathbb{R}$ a nechť pro $f \in \mathcal{P}_{2 \pi}$ existuje $f^\prime (x)$. Potom $f(x)=F_f(x)$.
\end{dusledek}

\begin{vetat}[Jordan-Dirichletovo kriterium - bez důkazu]
Nechť $f \in \mathcal{P}_{2\pi}$ je po částech monotónní. Tedy nechť existuje konečně mnoho bodů $0=a_1 < a_2 < \ldots < a_m = 2 \pi$ tak, že $f$ je monotónní na $(a_i, a_{i+1})$ pro $i \in \{1, \ldots, m-1 \}$. Potom Fourierova řada funkce $f$ konverguje v bodě $x$ k hodnotě $\frac{f(x+) + f(x-)}{2}$ pro všechna $x \in \mathbb{R}$.
\end{vetat}

\pagebreak
\setcounter{vety}{0}
\section{Základy komplexní analýzy}

Připomenutí vlastností $\mathbb{C}$ a operací $+$ a $\times$ na $\mathbb{C}$. Limita posloupnosti $z_n = a_n + i b_n \in \mathbb{C}$ je definována jako $\lim_{n \rightarrow \infty} b_n$, pokud obě limity reálných čísel existují.

\subsection{Holomorfní funkce a křivkový integrál}

\begin{definice}
Nechť $f$ je funkce definovaná na okolí bodu $z_0 \in \mathbb{C}$ a zobrazující do $\mathbb{C}$. Komplexní derivací $f$ v $z_0$ nazýváme komplexní číslo
$$f^\prime (z_0) = \lim_{z \rightarrow z_0} \frac{f(z) - f(z_0)}{z - z_0}$$
pokud tato limita existuje.
\end{definice}

\begin{definice}
Nechť $G \subset \mathbb{C}$ je otevřená. Funkce $f : G \rightarrow C$ se nazývá \emph{holomorfní}, má-li ve všech bodech $G$ komplexní derivaci.
\end{definice}

\begin{poznamka}
Jsou-li $f$ a $g$ holomorfní na $G$, pak jsou $f+g$ i $f g$ holomorfní na $G$ a $f/g$ je holomorfní na $G \cap \{g \neq 0\}$.
\end{poznamka}

\begin{definice}
Zobrazení $\varphi : [a,b] \rightarrow \mathbb{C}$ je křivka a $f : \langle \varphi \rangle \rightarrow \mathbb{R}$ je spojité zobrazení. Definujeme křivkový integrál
$$\int_{\langle \varphi \rangle} f(z) dz = \int_\alpha^\beta f( \varphi (t)) \varphi^\prime(t) dt$$
existuje-li integrál na pravé straně. Tento integrál můžeme značit i $\int_\varphi f(z) dz$.
\end{definice}

\begin{vetat}[Cauchyho věta pro trojúhélník]
Nechť $f$ je holomorfní na otevřené množině $G \subset \mathbb{C}$ a $\Delta \subset G$ je trojúhelník. Pak $\int_{\delta \Delta} f(z) dz = 0$.
\end{vetat}

\begin{vetabd}[Cauchy]
Nechť $f$ je holomorfní na otevřené množině $G \subset \mathbb{C}$. Nechť $\langle \varphi \rangle \subset G$ je uzavřená křivka taková, že vnitřek $\langle \varphi \rangle \subset G$ (tedy případné "díry" uvnitř G nejsou uvnitř $\langle \varphi \rangle$). Pak $\int_\varphi f(z)dz=0$.
\end{vetabd}

\begin{vetal}[Cauchyův vzorec]
Nechť $f$ je holomorfní na kruhu $B(z_0, R)$ a $0<r<R$. Pro křivku $\varphi(t) = z_0 + r e^{it}$, $t \in [0,2\pi]$, platí

\begin{equation}
\frac{1}{2 \pi i} \int_\varphi \frac{f(z)}{z-s}  = \left\{ \begin{array}{ll}
 f(z) & \textrm{pro $|s-z_0| < r$} \nonumber\\
 0 & \textrm{pro $| s-z_0 | > r$}
  \end{array} \right.
\end{equation}
\end{vetal}

\begin{vetat}[Liouville]
Nechť $f$ je holomorfní a omezená na $\mathbb{C}$. Pak $f$ je konstantní.
\end{vetat}

\begin{vetal}[Základní věta algebry]
Každý nekonstantní polynom (s komplexními koeficienty) má v $\mathbb{C}$ alespoň jeden kořen.
\end{vetal}

\subsection{Rozvoj do Taylorovy a Laurentovy řady}

\begin{definice}
Nehcť $z_0 \in \mathbb{R}$ a $a_n \in \mathbb{C}$ pro $n \in \mathbb{N}_0$. Řadu funkcí $\sum_{n=0}^{\infty} a_n (z-z_0)^n$ pro $z \in \mathbb{C}$ nazýváme mocninnou řadou s koeficienty $a_n$ o středu $z_0$.
\end{definice}

\begin{vetat}[o rozvoji do Taylorovy řady]
Nechť $f$ je holomorfní na kruhu $B(z_0, R)$. Pak existuje právě jedna mocninná řada s poloměrem konvergence alespoň $R$, že na $B(z_0, R)$ platí
$$f(z) = \sum_{n=0}^\infty a_n (z-z_0)^n$$
Navíc platí $a_n = \frac{f^{(n)}(z_0)}{n!}$ pro všechna $n \in \mathbb{N}_0$.
\end{vetat}

Jako u reálných mocninných řad lze na kruhu konvergence prohazovat $\sum$ a derivaci a důkaz je podobný.

\begin{dusledek}
Je-li $f$ holomorfní na $G$, pak na $G$ existují derivace všech řádů $f^{(k)}$ pro $k \in \mathbb{N}$.
\end{dusledek}

\begin{definice}
Množina $G \subset \mathbb{C}$ se nazývá \emph{oblast}, pokud je otevřená a souvislá. Tedy pokud platí
$$\forall A, B \in G \textrm{ otevrene v } G, G=A \cup B, A \cap B = \emptyset \Rightarrow A = \emptyset \textrm{ nebo } B = \emptyset$$
\end{definice}

\begin{vetal}[o jednoznačnosti holomorfní funkce]
Nechť $G \subset \mathbb{C}$ je oblast a $f, g$ jsou holomorfní na $G$. Předpokládejme, že množina
$$M = \left\{ z \in G : f(z)=g(z) \right\} $$
má hromadný bod v $G$, neboli existují $z_n \in M$ a $z_0 \in G$ takové, že $z_n \stackrel{n \rightarrow \infty}{\rightarrow} z_0$. Pak $f=g$ na $G$.
\end{vetal}

\begin{definice}
Řekneme, že funkce $f$ má v bodě $z_0$ pól násobnosti nejvýše $k \in \mathbb{N}$, je-li funkce 
\begin{equation}
F(z) = \left\{ \begin{array}{ll}
 (z-z_0)^{k+1}f(z) & \textrm{pro $z \neq z_0$} \nonumber\\
 0 & \textrm{pro $z=z_0$}
  \end{array} \right.
\end{equation}

holomorfní na nějakém okolí bodu $z_0$. Řekneme, že má pól násobnosti $k$, je-li $k \in \mathbb{N}$ nejmenší s touto vlastností.
\end{definice}

Například funkce $f(z) = 1 / z^k$ má v bodě 0 pól násobnosti k.

\begin{definice}
Nechť $M \subset G \subset \mathbb{C}$ je konečná množina. Řekneme, že funkce $f : G \backslash M \rightarrow \mathbb{C}$ je \emph{meromorfní} v $G$, pokud je $f$ holomorfní na $G \backslash M$ a v bodech $M$ má $f$ póly (konečné násobnosti).
\end{definice}

\begin{vetat}[o rozovji do Laurentovy řady]
Nehcť $f$ je holomorfní na mezikruží $B(z_0, R) \backslash \overline{B(z_0, r)}$, $0 < r < R$. Pak existují jednoznačně určená čísla $a_k \in \mathbb{C}$, $k \in \mathbb{Z}$, že platí 
$$f(z) = \sum_{k= - \infty}^\infty a_k (z-z_0)^k \textrm{ pro všechna } z \in B(z_0, R) \backslash \overline{B(z_0, r)}$$
\end{vetat}

\subsection{Reziduová věta a její aplikace}

\begin{definice}
Nechť $\sum_{k =- \infty}^\infty a_k ( z-z_0)^k$ je Laurentova řada funkce $f$. \emph{Rezuduum} funkce $f$ v bodě $z_0$ nazveme koeficient $a_{(-1)}$ a značíme ho $res_{z_0} f$.
\end{definice}

\begin{definice}
\emph{Index bodu $z_0$} vzhledem v uzavřené křivce $\varphi$ je definován jako
$$ind_\varphi z_0 = \frac{1}{2 \pi i} \int_\varphi \frac{1}{z-z_0}dz$$
\end{definice}

Index bodu udává, kolikrát oběhne křivka $\varphi$ okolo bodu $z_0$, pokud uvažujeme násobnost a obíhání v opačném směru bereme s opačným znaménkem.

\begin{vetat}[Reziduová věta]
Nechť $G \subset \mathbb{C}$ je oblast, $f$ je meromorfní funkce na $G$, $\varphi$ je křivka a póly $f$ neleží na $\langle \varphi \rangle (\subset G)$. Pak platí
$$\int_\varphi f(z) dz = 2 \pi i \sum_{\{z: z \textrm{ je pól } f\}} res_z f int_z f$$
\end{vetat}

\begin{proof}
Označme $P = \left\{ z \in G \textrm{ : } f(z)=+\infty \textrm{ resp. $f$ má pól v $z$} \right\}$. 
Pro $z_0 \in G$ označme 
$$H_{z_0} = \sum_{k = -kz}^{-1} a_k (z-z_0)^k$$ 
část rozvoje $f$ do Laurentovy řady.
Pak $$F(z) = f(z) - \sum_{z_0 \in P} H_{z_0}(z)$$ je $F$ holomorfní na $G$.
Podle Cauchyovy věty $\int_\varphi F(z) dz = 0$. Tedy
\begin{eqnarray}
\int_\varphi F(z) dz & = & \int_\varphi \sum_{z_0 \in P} H_{z_0}(z) dz \nonumber\\
& = & \sum_{z_0 \in P} \sum_{k=-kz}^{-1} \int_\varphi a_k (z-z_0)^k dz \nonumber\\
& = & \sum_{z_0 \in P} res_{z_0} f \int_\varphi \frac{1}{z-z_0} dz \nonumber\\
& = & 2 \pi i \sum_{z_0 \in P} res_{z_0} f ind_\varphi z_0 \nonumber
\end{eqnarray}
\end{proof}

\end{document}