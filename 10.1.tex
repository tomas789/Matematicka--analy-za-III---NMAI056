\begin{definice}
Nechť $J \subset \mathbb{R}$ je interval a nechť máme $f : J \rightarrow \mathbb{R}$ a $f_n : J \rightarrow \mathbb{R}$ pro $n \in \mathbb{N}$. Řekneme, že posloupnost funkcí $\{f_n\}$:

\begin{enumerate}
\item \emph{konverguje bodově} k $f$ na $J$, pokud pro každé $x \in J$ platí $\lim_{n \rightarrow \infty} f_n(x) = f(x)$, neboli
$$\forall x \in J \forall \epsilon > 0 \exists n_0 \in \mathbb{N} \forall n \geq n_0 : |f_n(x) - f(x)| < \epsilon$$
Značíme $f_n \rightarrow f$ na $J$.
\item \emph{konverguje stejnoměrně} k $f$ na $J$, pokud
$$\forall \epsilon > 0 \exists n_0 \in \mathbb{N} \forall n \geq n_0 \forall x \in J : | f_n(x) - f(x) | < \epsilon$$
Značíme $f_n \con f$.
\item \emph{konverguje lokálně stejnoměrně}, pokud pro každý omezený uzavřený interval $[a, b] \subset J$ platí: $f_n \rightrightarrows f$ na $[a, b]$. Značíme $f_n \conloc f$
\end{enumerate}
\end{definice}

\begin{vetal}[kritérium stejnoměrné konvergence]
Nechť $f_n, f:J \rightarrow \mathbb{R}$ pak
$$f_n \con f na J \Leftrightarrow \lim_{n \rightarrow \inf} \sup \{ |f_n(x) = f(x)|; x \in J \} = 0$$
\end{vetal}
\begin{dukaz}
$f_n \rightrightarrows f \leftrightarrow \forall \epsilon > 0 \exists n_0 \in \mathbb{N} \forall n \qeg n_0 \forall x \in J: |f_n(x) - f(x)| < \epsilon$
$\Leftrightarrow$
$\forall \epsilon > 0 \exists n_0 \in \mathbb{N} \forall n \geq n_0 : \sup \{|f_n(x) - f(x)|; x\in J} \leq \epsilon$
$\Leftrightarrow$
$\lim_{n \to infty} sup{|f_n(x)-f(x)|;x \in J} = 0$
$\Box$
\end{dukaz}

\begin{vetat}[Bolzano-Cauchyho podmínka pro stejnoměrnou konvergenci]
Nechť $f_n,f : J \rightarrow \mathbb{R}$. Pak
$$f_n \con f na J \Leftrightarrow \forall \epsilon > 0 \exists n_0 \forall m,n \geq n_0 \forall x \in J : | f_n(x) - f_m(x)| < \epsilon$$
\end{vetat}
\begin{dukaz}
"$\Rightarrow$" 
Nechť $\epsilon > 0.$ Zvolme $n_0$ z def. $f_n \rightrightarrows f.$ 
Nyní $\forall x \in J \exists m,n \geq n_0 |f_n(x) - f_m(x) | \leq |f_n(x) - f(x)| + |f(x) - f_m(x)| < \epsilon + \epsilon $
"$\Leftarrow$"
Nechť $x \in J$ je pevné. Platí $\forall \epsilon > 0 \exists n_0 \in mathbb{N} \forall m,n \geq n_0|f_n(x) - f_m(x)| < \epsilon$
Tedy posloupnosť ${f_n(x)}_{n=1}^\infty$ splňuje BC podmínku pro konvergenci posloupnosti Teda existuje její limita, zn.: $f(x)$.
Víme $\forall \epsilon > 0 \exists n_0 \in \mathbb{N} \forall m,n \geq n_0 \forall x \in J: |f_n(x)-f_m(x)|<\epsilon$
Provedme $lim_{n \to \infty}$, dostaneme:
$\forall \ epsilon > 0 \rightarrow n_0 \in \mathbb{N} \forall n \geq n_0 \forall x\ in J: |f_n(x) - f(x)| \leq \epsilon$
$\Rightarrow$
$f_n \rightrightarrow f$
$\Box$
\end{dukaz}

\begin{vetat}[Moore-Osgood]
Nechť $x_0$ je krajní bod intervalu J (může být i $\pm \infty$). Nechť $f, f_n : J \rightarrow \mathbb{R}$ splňují
\begin{enumerate}
\item $f_n \con f$ na $J$,
\item existuje $\lim_{x \rightarrow x_0} f_n(x) = a_n \in \mathbb{R}$ pro všechna $n \in \mathbb{N}$
\end{enumerate}
Pak existují $\lim_{n \rightarrow \infty} a_n$ a $\lim_{x \rightarrow x_0} f(x)$ a jsou si rovny, neboli:
$$\lim_{n \rightarrow \infty} \lim_{x \rightarrow x_0} f_n(x) = \lim_{x \rightarrow x_0} \lim_{n \rightarrow \infty} f_n(x)$$
\end{vetat}

\begin{dusledek}
Nechť $f_n \con f$ na $I$ a nechť $f_n$ jsou na $I$ spojité. Pak $f$ je spojitá na $I$.
\end{dusledek}

\begin{vetal}[o záměně limity a integrálu]
Nechť funkce $f_n \con f$ na $[a,b]$ a nechť $f_n \in \mathbb{R} ([a,b])$. Pak $f \in \mathbb{R}([a,b])$ a 
$$(R) \int_a^b f(x) dx = \lim_{n \rightarrow \infty} (R) \int_a^b f_n(x) dx$$
\end{vetal}

\begin{vetat}[o záměně limity a derivace]
Nechť funkce $f_n$, $n \in \mathbb{N}$, mají vlastní derivaci na intervalu $(a,b)$ a nechť:
\begin{enumerate}
\item existuje $x_0 \in (a,b)$ tak, že $\{f_n(x_0)\}_{n=0}^{\infty}$ konverguje,
\item pro derivace $f_n'$ platí $f_n' \conloc$ na $(a,b)$
\end{enumerate}
Potom existuje funkce $f$ tak, že $f_n \conloc f$ na $(a,b)$, $f$ má vlastní derivaci a platí $f_n' \conloc f'$ na $(a,b)$.
\end{vetat}
