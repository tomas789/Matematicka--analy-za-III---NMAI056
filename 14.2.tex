\begin{definice}
Nechť $(P, \varrho)$ je metrický prostor a nechť $x_n \in P, n \in \mathbb{N}$, je posloupnost bodů z $P$. Posloupnost $\{ x_n \}_{n \in \mathbb{N}}$ nazveme \emph{cauchyovskou}, pokud
$$\forall \varepsilon > 0 \ \exists n_0 \in \mathbb{N} \ \forall m,n \geq n_0 \textrm{ : } \varrho(x_n, x_m) < \varepsilon$$
Posloupnost $\{ x_n \}_{n \in \mathbb{N}}$ nezveme \emph{konvergentní}, pokud existuje $x \in P$ tak, že
$$\lim_{n \rightarrow \infty} \varrho (x_n, x) = 0$$
Řekneme, že $(P, \varrho)$ je \emph{úplný}, pokud je každá cauchzovská poslouopnost konvergentní.
\end{definice}

\begin{vetal}[úplnost $\mathbb{R}^n$]
Metrický prostor $(\mathbb{R}, |.|)$ je úplný.
\end{vetal}

\begin{priklad}
\begin{enumerate}
\item Metrický prostor $(Q, |.|)$ není úplný.
\item Metrický prostor všech spojitých funkcí $C([0, 1])$ s metrikou
$$\varrho_1(f,g) = (R) \int_0^1 |f(x) - g(x)| dx$$
není úplný.
\item Metrický prostor všech Lebesqueovsky integrovatelných funkcí $L([0,1])$ s metrikou
$$\varrho(f,g) = (L) \int_0^1 |f(x)-g(x)|dx$$
je úplný
\end{enumerate}
\end{priklad}

\begin{vetat}[úplnost spojitých funkcí]
Metrický prostor spojitých funkcí $(C([0,1]), \varrho)$ se supremovou metrikou
$$\varrho(f,g) = \sup_{x \in [0,1]} |f(x) - g(x)|$$
je úplný
\end{vetat}

\begin{vetat}[Banachova věta o kontrakci]
\label{Banachova věta o kontrakci}
Nechť $(P, \varrho)$ je úplný metrický prostor a $K < 1$. Nechť $T \textrm{ : } P \rightarrow P$ je zobrazení takové, že 
$$\varrho(Tx, Ty) \leq K \varrho(x,y) \quad \forall x,y \in P$$
Pak existuje právě jeden bod $x_0 \in P$ tak, že $T(x_0)=x_0$
\end{vetat}

\begin{proof}
\textbf{jednoznačnost: } Nechť $T(x_0) = x_0$ a $T(\tilde{x}_0) = \tilde{x}_0$, $x_0 \neq \tilde{x}_0$. Pak 
$$K \varrho (x_0, \tilde{x}_0) \geq \varrho (T(x_0),T(\tilde{x}_0)) = \varrho (x_0, \tilde{x}_0) \Rightarrow K \geq 1$$
a to je spor.

\textbf{existence: } Zvolme $x_1 \in P$ libovolně a položme $x_{k+1} = T(x_k)$. Tvrdíme, že je cauchyovská.
$$\varrho( x_{k+1}, x_k ) = \varrho (T(x_k), T(x_{k-1})) \leq K \varrho (x_k,x_{k-1}) \leq K^2 \varrho (x_{k-1},x_{k-2}) \leq \ldots \leq K^{k-1} \varrho (x_2, x_1)$$

Nechť $m,n \geq n_0$ a navíc $m > n$
$$\varrho (x_m, x_n) \leq \varrho (x_m, x_{m-1}) + \varrho (x_{m-1}, x_{m-2}) + \ldots + \varrho (x_{n+1}, x_n) $$
$$\leq K^{m-2} \varrho (x_2, x_1) + \ldots + K^{n-1} \varrho (x_2, x_1) \leq K^{n-1} \varrho (x_2, x_1) \frac{1}{1-K} \overset{x \to \infty}{\to} 0$$

Z toho dostaneme cauchyho vlastnost $x_2$. Z úplnosti $P$ $\exists x_0 \textrm{ : } x_n \overset{P}{\to} x_0$. Tvrdím $T(x_0)=x_0$
\begin{eqnarray*}
x_{n+1} & = & T(x_n) \\
\lim_{n \to \infty} \quad x_0 & = & T(x_0)
\end{eqnarray*}

$x_n \to x_0 \textrm{ : } \varrho (x_m, x_0) \to 0$, $\varrho (T(x_n), T(x_0)) \leq K \varrho (x_n, x_0) \to 0$
\end{proof}

\begin{poznamka}
Věta T\ref{Banachova věta o kontrakci} se používá například v \emph{fractal compression}. Touto metodou se však dosahuje horších výsledků než například konpresí \emph{JPEG} a proto se v praxi nepoužívá.
\end{poznamka}



\begin{vetat}[o zúplnění metrického prostoru]
Nechť $(Q, \varrho)$ je metrický prostor. Pak existuje úplný metrický prostor $(P, \sigma)$ tak, že $Q \subset P$ a 
$$\sigma(x,y) = \varrho(x,y) \quad \forall x,y \in Q$$
\end{vetat}

\begin{proof}
Položme $P = C(Q)$ s metrikou $\varrho (f,g) = \sup_{y \in Q} | f(x)-g(x) |$. 

\begin{poznamka} 
Důkaz je analogický a důkazu věty o úplnosti $C([0,1])$
\end{poznamka}

$x \in Q \textrm{ <$\cdots$> } f_x(y) = \varrho(x,y) \in C(Q)$. 

Pro $x \neq z$ a $f_x \neq f_z$ chceme $\sigma (x,z) = \varrho (f_x, f_z)$
$$\varrho (f_x, f_z) = \sup_{y \in Q} | \sigma (x,y) - \sigma (z,y) | \overset{\textrm{$\triangle$ nerovnost}}{\leq} \sigma (x,z) \Rightarrow \varrho (f_x, f_z) = \sigma(x,z)$$ 
\end{proof}


\begin{vetal}[úplnost a uzavřená podmnožina]
Nechť $(P, \varrho)$ je úplný metrický prostor a $F \subset P$ je uzavřená podmnožina. Pak je metrický prostor $(F, \varrho)$ úplný.
\end{vetal}