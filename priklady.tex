\begin{multicols}{2}

Vyšetřete stejnoměrnou a lokálně stejnoměrnou konvergenci posloupnosti funkcí $f_n$ na intervalu $(0,1)$
$$f_n(x) = x^n - x^{3n}$$

\separator

Vyšetřete stejnoměrnou a lokálně stejnoměrnou konvergenci posloupnosti funkcí $f_n$ na intervalu $(-\infty,0)$
$$f_n(x) = e^{nx}$$

\separator

Vyšetřete stejnoměrnou a lokálně stejnoměrnou konvergenci funkcí na $\mathbb{R}$
$$\sum_{n=1}^\infty \sin \left( \frac{1}{x^2 + n^2} \right)$$

\separator

Určete poloměr konvergence následující řady
$$\sum_{n=1}^\infty 2^n x^{4^n}$$

\separator

Sečtěte řadu
$$\sum_{n=1}^{\infty} \frac{3^{-n}}{n}$$

\separator 

Najděte Fourierovu řadu následující funkce.
$$f(x) = (\textrm{sgn} x) \sin x \quad \textrm{na intervalu} \ [-\pi, \pi)$$

\separator

Najděte Fourierovu řadu následující funkce a zjistěte k jakým hodnotám konverguje
$$f(x)=x \textrm{ pro } x \in [0,\pi) \textrm{ a } f(x)=0 \textrm{ pro } x \in [-\pi,0)$$

\separator

Spočtěte integrál podle definice křivkového integrálu
$$\int_{S^1} \sqrt{z} + \frac{1}{\sqrt{z}} dz$$
kde $S^1$ je jednotková kružnice probíhaná v kladném smzslu a volíme $\sqrt{1}=1$

\separator

Najděte residua následující funkce
$$f(z)=\frac{z^2}{(1+z^2)^2}$$

\separator

Spočtěte
$$\int_{-\infty}^\infty \frac{1}{(x^2 + 9)^3} dx$$

\separator

Vyšetřete bodovou a stejnoměrnou konvergenci následujících funkcí
\begin{enumerate}
\item $f_n(x) = \frac{x^n}{1+x^n}$ na $[0,1]$
\item $f_n(x) = x^n - x^{2n}$ na $(0,1)$
\item $f_n(x) = \sin \left( \frac{x}{n} \right)$ na $(-100,100)$
\item $f_n(x) = n \left( \sqrt{x + \frac{1}{n}} - \sqrt{x} \right)$ na $(0,\infty)$
\item $f_n(x) = \sqrt[n]{1+x^n}$ na $(0,\infty)$
\item $f_n(x) = n \left( x^\frac{1}{n} - 1 \right)$ na $(1,100)$ a $(1,\infty)$
\end{enumerate}

\separator

Zjistěte, zda $f_n$ jdou stejnoměrně a lokálně stejnoměrně k nějaké $f$ na intervalu $J$
\begin{enumerate}
\item $f_n(x) = e^{n (x-1)}$, $J = (0,1)$
\item $f_n(x) = \sin (\pi x^n)$, $J = [0,1)$
\item $f_n(x) = \frac{nx}{1+n+x}$, $J = (0,\infty)$
\item $f_n(x) = n x e^{-nx^2}$, $J = \mathbb{R}$
\item $f_n(x) = \frac{\ln (nx)}{n}$, $J = (0,\infty)$
\item $f_n(x) = \left( 1 + \frac{x}{n} \right)^n$, $J = (0,\infty)$
\item $f_n(x) = x \arctan (nx)$, $J = \mathbb{R}$
\item $f_n(x) = x^{2n} - x^{3n}$, $J = (0,1)$
\item $f_n(x) = \sqrt{x^2 + \frac{1}{n^2}}$, $J = \mathbb{R}$
\item $f_n(x) = \sqrt{x} \sin \frac{x}{n}$, $J = \mathbb{R}$
\end{enumerate}

\separator

Zjistěte na kterách intervalech konverguje řada stejnoměrně resp. lokálně stejnoměrně.
\begin{enumerate}
\item $\sum_{n = 1}^\infty x^n$
\item $\sum_{n = 1}^\infty \frac{1}{n^2 x^2 + 1}$
\item $\sum_{n = 1}^\infty x^2 e^{-nx}$
\item $\sum_{n = 1}^\infty \ln \left( 1 + \frac{x^2}{n \ln^2 n } \right)$
\item $\sum_{n = 1}^\infty \arctan \left( \frac{2x}{x^2 + n^3} \right)$
\item $\sum_{n = 1}^\infty \frac{nx}{(1+x)(1+2x)\ldots(1+nx)}$ na $(0,\infty)$
\end{enumerate}

\separator

Zjistěte, zda $\sum_{n=0}^{\infty}$ je na $(a,b)$ diferencovatelná a zda konverguje lokálně stejnoměrně
\begin{enumerate}
\item $\sum_{n=1}^\infty \frac{1}{n^2 + x}$ na $(-1, \infty)$
\item $\sum_{n=1}^\infty (1-x^n) x^n$ na $[0,1]$
\item $\sum_{n=1}^\infty \frac{x^n}{n^s}$ na $[0, \infty)$ s parametrem $s \in \mathbb{R}$
\item $\sum_{n=1}^\infty \frac{\arctan \left( \frac{x}{n} \right)}{n}$ na $(-1, \infty)$
\item $\sum_{n=1}^\infty \frac{\sin \frac{x}{\sqrt{n}}}{x^2 + n}$ na $\mathbb{R}$
\item $\sum_{n=1}^\infty \sin \frac{1}{x^2 + n^2}$
\item $\sum_{n=1}^\infty 2^n \arctan ( 3^n x^n )$ na $[0,1]$
\end{enumerate}

\separator

Sečtěte řadu
\begin{enumerate}
\item $x - 4x^2 + 9x^3 - \ldots$
\item $\sum_{n=0}^\infty \frac{x^{4n+1}}{4n+1}$
\item $\sum_{n=0}^\infty \frac{n^2}{2^n}$
\end{enumerate}

\separator

Určete poloměr konvergence
\begin{enumerate}
\item $\sum_{n=1}^\infty \frac{z^n}{n^3}$
\item $\sum_{n=1}^\infty \frac{(n!)^2}{(2n)!} z^n$
\item $\sum_{n=1}^\infty \frac{z^{n!}}{n!}$
\item $\sum_{n=1}^\infty \left( na^n + \frac{b^n}{n^2} \right) z^n$ pro $0 < a < b$
\item $\sum_{n=1}^\infty \frac{n!}{a^{n^2}} z^n$ pro $a > n$
\end{enumerate}

\separator

Spočtěte $\int_C \frac{1}{\sqrt{z}} dz$, kde $C$ je jednotková kružnice probíhající v kladném smyslu

\separator

Spočtěte $\int_C z dz$ po oblouku paraboly $y = 1-x^2$ od bodu $[1,0]$ do bodu $[-1,0]$

\separator

Spočtěte $\int_C \frac{z e^z}{z^2 + 4} dz$ kde
\begin{itemize}
\item $C$ je kružnice $S(2i, 2)$
\item $C$ je kružnice $S(0,10)$
\end{itemize}

\separator 

Spočtěte $\int_C z^a dz$, kde $C$ je jednotková kružnice probíhající v kladném smyslu a $a \in \mathbb{R}$

\separator

Spočtěte $\frac{1}{2 \pi i} \int_C \frac{e^z}{z^4 - 1} dz$, kde
\begin{itemize}
\item $C = S(1,1)$
\item $C = S \left(-2,\frac{1}{2} \right)$
\end{itemize}

\separator

Spočtěte $\int_C \frac{e^z}{z^2 - a^2} dz$ kde parametr $a \in \mathbb{R} \backslash \{ 1, 3 \}$ a $C = S(2,1)$

\separator

Spočtěte
\begin{enumerate}
\item $\int_0^\infty \frac{1}{(x^2 + 1)^3} dx$
\item $\int_0^\infty \frac{x^2 + 1}{x^4 + 1} dx$
\item $\int_{-\infty}^\infty \frac{1}{x^6 + 1} dx$
\item $\int_0^\infty \frac{x^2 + 1}{x^4 + 1} dx$
\end{enumerate}
 
\separator

Najděte rezidua následujících funkcí
\begin{enumerate}
\item $f(z) = \frac{e^z}{z^2 (z^2 + 9)}$
\item $f(z) = \frac{1}{z^3 - z^5}$
\item $f(z) = \frac{\sin (2z)}{(1+z)^3)}$
\end{enumerate}

\end{multicols}
