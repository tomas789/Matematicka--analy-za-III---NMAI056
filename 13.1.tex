\begin{definice}
Nechť $f$ je funkce definovaná na okolí bodu $z_0 \in \mathbb{C}$ a zobrazující do $\mathbb{C}$. Komplexní derivací $f$ v $z_0$ nazýváme komplexní číslo
$$f^\prime (z_0) = \lim_{z \rightarrow z_0} \frac{f(z) - f(z_0)}{z - z_0}$$
pokud tato limita existuje.
\end{definice}

\begin{definice}
Nechť $G \subset \mathbb{C}$ je otevřená. Funkce $f : G \rightarrow C$ se nazývá \emph{holomorfní}, má-li ve všech bodech $G$ komplexní derivaci.
\end{definice}

\begin{poznamka}
Jsou-li $f$ a $g$ holomorfní na $G$, pak jsou $f+g$ i $f g$ holomorfní na $G$ a $f/g$ je holomorfní na $G \cap \{g \neq 0\}$.
\end{poznamka}

\begin{definice}
Zobrazení $\varphi : [a,b] \rightarrow \mathbb{C}$ je křivka a $f : \langle \varphi \rangle \rightarrow \mathbb{R}$ je spojité zobrazení. Definujeme křivkový integrál
$$\int_{\langle \varphi \rangle} f(z) dz = \int_\alpha^\beta f( \varphi (t)) \varphi^\prime(t) dt$$
existuje-li integrál na pravé straně. Tento integrál můžeme značit i $\int_\varphi f(z) dz$.
\end{definice}

\begin{vetat}[Cauchyho věta pro trojúhélník]
Nechť $f$ je holomorfní na otevřené množině $G \subset \mathbb{C}$ a $\Delta \subset G$ je trojúhelník. Pak $\int_{\delta \Delta} f(z) dz = 0$.
\end{vetat}

\begin{proof}
Sporem : nechť $\int_{\partial \triangle} f = M > 0$. Rozdělíme na čtyři trojúhelníky.

% TODO doplnit obrázek

Sečteme $\int$ přes hranice $\triangle_1$, $\triangle_2$, $\triangle_3 + \triangle_4$ v opačném směru.

Platí 
$$\int_{\partial \triangle_4} f = \int_{\partial \triangle_1} f + \int_{\partial \triangle_2} f + \int_{\partial \triangle_3} f + \int_{\partial \triangle_4} f$$

Tady $\exists \triangle_i \textrm{ : } \left| \int_{\partial \triangle_i} f \right| \geq \frac{M}{4}$. Tento trojúhelník opět rozdělíme na čtyři kusy. Dostaneme posloupnost trpjúhelníků $\triangle^K$ tak, že 
$$\left| \int_{\partial \triangle^K} f \right| \geq \frac{M}{4^K}$$
a
$$(\textrm{obvod $\triangle^K$}) = \frac{\textrm{obvod $\triangle$}}{2^K}$$

$\triangle^K$ uzavřené, zanořené do sebe $\Rightarrow$ 
$$\exists x_0 \in \bigcap_{K=1}^{\infty} \triangle^K$$
(důkaz přes konvergentní Cauchyovskou posloupnost)

Funkce $f$ je diferencovatelná v $x_0$, tady $f(z) = f(z_0) + f^\prime(z_0)(z-z_0) + \varepsilon (z-z_0)(z-z_0)$
$$\lim_{z \to z_0} \varepsilon (z-z_0) = 0$$

$$\left| \int_{\partial \triangle^K} f(z) dz \right| = \left| \int_{\partial \triangle^K} \underbrace{f(z_0) + f^\prime (z_0)(z-z_0)}_\circledast + \varepsilon (z-z_0) (z-z_0) dz \right|$$

$\circledast$ : má primitivní funkci $f(z)z + f^\prime(z_0) \frac{(z-z_0)^2}{2}$ a $\partial \triangle_K$ je uzavřená křivka, tedy 
$$\int_{\partial \triangle^K} \left[ f(z_0) + f^\prime (z_0) (z-z_0) \right] = 0$$

$$\frac{M}{4^K} \leq \left| \int_{\partial \triangle^K} f(z) dz \right| = \left| \int_{\partial \triangle^K} \varepsilon (z-z_0) (z-z_0) dz \right| \leq \textrm{délka } \partial \triangle_K \sup_{z \in \partial \triangle_K} \left| \varepsilon (z-z_0) \right| \sup_{z \in \partial \triangle_K} \left| z-z_0 \right| $$
$$ \leq \frac{c}{2^K} \varepsilon_K \frac{c}{2^K} \Rightarrow M \leq c^2 \varepsilon_K \overset{k \to \infty}{\rightarrow} 0 \Rightarrow M \leq 0$$

a to je spor.
\end{proof}

\begin{vetabd}[Cauchy]
Nechť $f$ je holomorfní na otevřené množině $G \subset \mathbb{C}$. Nechť $\langle \varphi \rangle \subset G$ je uzavřená křivka taková, že vnitřek $\langle \varphi \rangle \subset G$ (tedy případné "díry" uvnitř G nejsou uvnitř $\langle \varphi \rangle$). Pak $\int_\varphi f(z)dz=0$.
\end{vetabd}

\begin{vetal}[Cauchyův vzorec]
Nechť $f$ je holomorfní na kruhu $B(z_0, R)$ a $0<r<R$. Pro křivku $\varphi(t) = z_0 + r e^{it}$, $t \in [0,2\pi]$, platí

\begin{equation}
\frac{1}{2 \pi i} \int_\varphi \frac{f(z)}{z-s}  = \left\{ \begin{array}{ll}
 f(z) & \textrm{pro $|s-z_0| < r$} \nonumber\\
 0 & \textrm{pro $| s-z_0 | > r$}
  \end{array} \right.
\end{equation}
\end{vetal}

\begin{proof}
\begin{enumerate}
\item $|s-z_0| > r$

pak $\frac{f(z)}{z-s}$ je holomorfní na $B(z_0, r + \varepsilon)$ dle Cauchyho věty

$$\int_\varphi \frac{f(z)}{z-s} dz = 0$$

\item $|s-z_0| < r$

Definujme funkce 
\begin{equation}
F(z)  = \left\{ \begin{array}{ll}
 \frac{f(z)-f(s)}{z-s} & \textrm{pro $z \neq s$} \nonumber\\
 f^\prime(s) & \textrm{pro $z=s$}
  \end{array} \right.
\end{equation}

Pak $F(z)$ je holomorfní na $B(z_0, R) \backslash \{ s \}$ a v $s$ spojitá. Dle Poznámky

$$\int_\varphi F(z) dz = 0 = \int_\varphi \frac{f(z)}{z-s} dz - \int_\varphi \frac{f(s)}{z-s} dz$$
\end{enumerate}
\end{proof}


\begin{vetat}[Liouville]
\label{Liouville}

Nechť $f$ je holomorfní a omezená na $\mathbb{C}$. Pak $f$ je konstantní.
\end{vetat}

\begin{vetal}[Základní věta algebry]
Každý nekonstantní polynom (s komplexními koeficienty) má v $\mathbb{C}$ alespoň jeden kořen.
\end{vetal}

\begin{proof}
Sporem. Nechť $\forall z \in \mathbb{C} \textrm{ : } P(z) \neq 0$. Pak $\frac{1}{P(z)}$ he holomorfní funkce na $\mathbb{C}$.

$$P(z) = a_n z^n + a_{n-1} z^{n-1} + \ldots + a_0 = a_n z^n \left( 1 + \frac{a_{n-1}}{a_n} \frac{1}{z} + \ldots + \frac{a_0}{a_n} \frac{1}{z^n} \right)$$

$\exists R > 0 \textrm{ } \forall |Z| > R \textrm{ : }$

$$\left( 1 + \frac{a_{n-1}}{a_n} \frac{1}{z} + \ldots + \frac{a_0}{a_n} \frac{1}{z^n} \right) > \frac{1}{2} \geq 1 - \frac{|a_{n-1}|}{|a_n|} \frac{1}{R} - \frac{|a_{n-2}|}{|a_n|} \frac{1}{R^2} - \ldots - \frac{|a_0|}{|a_n|} \frac{1}{R^n}$$

Tedy 
$$\left| \frac{1}{P(z)} \right| \leq \frac{1}{|a_n z^n| (\ldots)} \leq \frac{2}{|a_n| |z^n|}$$

Tedy $\frac{1}{P(z)}$ je omezená holomorfní funkce, tedy dle Věty \ref{Liouville} je $\frac{1}{P(z)}$ konstantní a to je spor.
\end{proof}
