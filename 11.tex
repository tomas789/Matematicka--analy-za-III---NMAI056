\begin{definice}
Nehcť $x_0 \in \mathbb{R}$ a $a_n \in \mathbb{R}$ pro $n \in \mathbb{N}_0$. Řadu funkcí $\sum_{n=0}^{\infty} a_n (x-x_0)^n$ nazýváme \emph{mocninnou řadou} s koeficienty $a_n$ o středu $x_0$.
\end{definice}

\begin{definice}
\emph{Poloměrem konvergence} mocninné řady $\sum_{n=0}^{\infty} a_n (x-x_0)^n$ nazveme $$R = \sup \left\{ r \in [ 0,\infty ) : \sum_{n=0}^{\infty}  a_n ( x - x_0 )^n \textrm{ konverguje } \forall x \in [ x_0 - r; x_0 + r ] \right\}$$
\end{definice}

\begin{vetal}[o poloměru konvergence mocninné řady]
Nechť $\sum_{n=0}^{\infty} a_n (x-x_0)^n$ je mocninná řada a $R \in [ 0, \infty ]$ její poloměr konvergence. Pak řada konverguje obsolutně pro všechna $x$ taková, že $| x - x_0| < R$ a diverguje pro všechna $x$ taková, že $| x - x_0 | > R$.
\end{vetal}

\begin{proof}
Nechť $|x-x_0|<R$, zvolme $r: |x-x_0|<r<R$. Z definice $R$, $\sum a_nr^n$ konverguje. Tedy je tato řada omezená, tedy $\exists K>0:\forall n \in \mathbb{N} |a_nr^n|<K$.
Nyní
$$|a_n(x-x_0|^n = |a_n r^n \frac{(x-x_0)^n}{r^n}| \leq |a_nr^n|. \left( \frac{|x-x_0|}{r} \right)^n \leq K \frac{|x-x_0|^n}{r^n} $$
$|x-x_o|<r$, tedy geometrická řada $\sum K. \left( \frac{|x-x_0|}{r} \right)^n$ konverguje. Ze srovnávacího kritéria $\sum a_n(x-x_o)^n$ konverguje.
Nechť $|x-x_o|>R$. Tvrdím, že $\sum a_n(x-x_o)^n$ diverguje. Jinak bychom našli $y: R< |y-x_0| < |x-x_0|$. Analogicky předchozí části důkazu:
$$\sum a_n(x-x_0)^n \textrm{ konverguje } \Rightarrow \sum a_n(y-x_0)^n \textrm{ absolutně konverguje } \rightarrow \textrm{ spor s definicí } R$$
\end{proof}

\begin{vetal}[výpočet poloměru konvergence]
Nechť $\sum_{n=0}^{\infty} a_n (x-x_0)^n$ je mocninná řada a $R \in [ 0, \infty ]$ její poloměr konvergence. Pak platí
$$R = \frac{1}{ \lim \sup_{n \rightarrow \infty} \sqrt[n]{| a_n |} }$$
Pokud existuje $\lim_{n \rightarrow \infty} \frac{|a_n|}{|a_{n+1}|}$, pak $R = \lim_{n \rightarrow \infty} \frac{|a_n|}{|a_{n+1}|}$.
\end{vetal}

\begin{proof}
Nechť $R = \frac{1}{\limsup \sqrt[n]{|a_n|}}$ a $0<R<\infty$ (pro $K = 0$ a $R = \infty$ analogicky)
Nechť $|x-x_0| < R$ , $\sum a_n(x-x_0)^n$ . Použijeme limitné odmocninové kritérium a dostaneme:
$$\limsup_{n \to \infty} \sqrt[n]{a_n(x-x_0)^n} = \limsup_{n \to \infty} \sqrt[n]{|a_n|} . |x-x_0| - \frac{1}{R} |x-x_0| < 1$$
Tedy řada konverguje.
Nechť $|x-x_0|>R$. Pak 
$$\limsup \sqrt[n]{a_n|x-x_0|} = \frac{1}{R}|x-x_0| > 1 \Rightarrow \limsup |a_n(x-x_0)|^n > 1$$
Tedy $\exists $ podposloupnosť, že $|a_{n_k} (x-x_0)^n| > 1$. Není splněna nutná podmínka konvergeci, tedy řada diverguje.
Nechť existuje $R = \lim_{n \to \infty } \frac{|a_n|}{a_{n+1}}$ a $|x-x_0|< R$. Podle limitního podílového kritéria:
$$\lim_{n \to \infty} \frac{|a_{n+1}(x-x_0)^{n+1}|}{a_n(x-x_0)^n|} = \lim_{n \to \infty} \frac{|a_{n+1}|}{a_n} |x-x_0| = \frac{1}{R} (x-x_0) < 1 \Rightarrow \sum a_n(x-x_0)^n \textrm{ konverguje.}$$
Pokud $|x-x_0| > R$ , pak 
$$\lim_{n \to \infty} \frac{|a_{n+1}(x-x_0)^{n+1}|}{a_n(x-x_0)^n|} = \frac{1}{R}|x-x_0|>1 \Rightarrow \textrm{řada diverguje.}$$
\end{proof}

\begin{vetal}[o stejnoměrné konvergenci mocniné řady]
Nechť $\sum_{n=0}^{\infty} a_n (x-x_0)^n$ je mocninná řada s poloměrem konvergence $R > 0$.  Pak řada konverguje lokálně stejnoměrně na $(x_0 - R, x_0 + R)$ (je-li $R=\infty$, pak na celém $\mathbb{R}$).
\end{vetal}

\begin{proof}
Nechť $r < R$, chceme $\sum \con \textrm{ na } [x_0-r,x_0+r]$. Z věty 1 víme absolutní konvergenci rady $\sum a_n.r^n$. Na $ \sum a_n(x-x_0)^n$ použijeme Weirstrassovo kritérium.
$$b_n = \sup_{x \in[x_0-r, x_0+r]} |a_n(x-x_0)^n| = |a_n|r^n \textrm{ a } \sum b_n \Rightarrow \sum a_n(x-x_0)^n \con \textrm{ na } [x_0-r, x_0+r]$$
\end{proof}

\begin{vetal}[o derivaci mocninné řady]
Nechť $\sum_{n=0}^{\infty} a_n (x-x_0)^n$ je mocninná řada s poloměrem konvergence $R > 0$. Pak $\sum_{n=1}^{\infty} n a_n (x-x_0)^{n-1}$ je také mocninná řada se stejným středem a poloměrem konvergence. Navíc pro $x \in ( x_0 - R, x_0 + R )$ ($\mathbb{R}$ pro $R = \infty$) platí
$$ \left( \sum_{n=0}^{\infty} a_n (x-x_0)^n \right)' = \sum_{n=1}^{\infty} n a_n (x-x_0)^{n-1}$$
\end{vetal}

\begin{proof}
Je vidět, že se jedná o mocninnou řadu se středem $x_0$ a koeficienty $\tilde{a}_n = (n+1)a_{n+1}$
$$R = \frac{1}{\limsup_{n \to \infty} \sqrt[n]{|a_n|}} \Rightarrow R = \frac{1}{\limsup_{n \to \infty} \sqrt[n]{n |a_n|}} \overset{\textrm{limita složené funkce}}{\Rightarrow} R = \frac{1}{\limsup_{n \to \infty} \sqrt[n-1]{n |a_n|}}$$

Tedy poloměr konvergence formální derivace je stejný. Dle předchozí věty na $(x_0 - R, x_0 + R) \textrm{ : } \sum n a_n (x-x_0)^{n-1} \conloc$, v $x=x_0$ konverguje, tedy mohu použít větu o derivování řad funkcí (Věta L\ref{o spojitosti a derivování řad funkcí}) a dostaneme
$$\left( \sum_{n=0}^\infty a_n (x-x_0)^n \right)^\prime = \sum_{n=1}^\infty n a_n (x-x_0)^{n-1}$$
\end{proof}


\begin{vetal}[o integrování mocninné řady]
Nehcť $\sum_{n=0}^{\infty} a_n (x-x_0)^n$ je mocninná řada s poloměrem konvergence $R > 0$. Pak $\sum_{n=0}^{\infty} \frac{a_n}{n+1} \left(x-x_0 \right)^{n+1}$ je také mocninná řada se stejným poloměrem konvergence. Navíc platí
$$\int \sum_{n=0}^{\infty} a_n (x-x_0)^n dx = \sum_{n=1}^{\infty} \frac{a_n}{n+1} \left( x-x_0 \right)^{n+1} + C  \quad \textrm{na } \left( x_0 - R, x_0 + R \right)$$
\end{vetal}

\begin{proof}
$$R = \frac{1}{\limsup \sqrt[n]{|a_n|}} \Rightarrow R = \frac{1}{\limsup \sqrt[n]{ \frac{|a_n|}{n+1}}} \Rightarrow R = \frac{1}{\limsup \sqrt[n+1]{ \frac {|a_n|}{n+1}}}$$
Tedy skutečně má řada $\sum \frac{a_n}{n+1}(x-x_0)^{n+1}$ stejný poloměr konvergence a střed.
Podle věty 3 $\sum a_n (x-x_0)^n $ konverguje $\conloc$ na $(x_0-R, x_0+R)$.
Podle věty o zámene lim a integrálu použité na částečné součty řady dostaneme: $\forall [c,d] \subset (x_0-R,x_0+R) $
$$\int_c^d \sum a_n(x-x_0)^n = \sum \int_c^d a_n(x-x_0)^n = \sum \frac{a_n}{n+1} \big[ (x-x_0)^{n+1} \big]_c^d $$
Funkce jsou spojité, z rovnosti integrálu plyne rovnost primitivních funkcí.
\end{proof}

\begin{vetat}[Abelova]
Nechť $\sum_{n=0}^{\infty} a_n (x-x_0)^n$ je mocninná řada s poloměrem konvergence $R > 0$. Nechť navíc $\sum_{n=0}^{\infty} a_n R^n$ konverguje. Potom řada $\sum_{n=0}^{\infty} a_n (x-x_0)^n$ konverguje stejnoměrně na $[ x_0, x_0 + R ]$ a platí
$$\sum_{n=0}^{\infty} a_n R^n = \lim_{r \rightarrow R_-} \sum_{n=0}^{\infty} a_n r^n$$
\end{vetat}

\begin{proof}
Předpokládejme bez újmy na obecnosti, že $x_0 = 0$. Označme $t_N = \sum_{n=N+1}^{\infty} a_n R^n$. Víme, že $\sum a_n R^n$ konverguje, tedy
$$\forall \varepsilon>0 \quad \exists n_0 \quad \forall n \geq n_0 \quad : \quad |t_N| < \varepsilon$$
\begin{eqnarray*}
a_n & = & a_n R^n \left( \frac{x}{R} \right)^n \\
& = & - t_N \left( \left( \frac{x}{R} \right)^n - \left( \frac{x}{R} \right)^{n+1} \right) + t_{n-1} \left( \frac{x}{R} \right)^n - t_n \left( \frac{x}{R} \right)^{n+1} 
\end{eqnarray*}
Sečteme od $N$ do $N+k$
$$\sum_{n=N}^{N+k} a_n x^n = \left[ \sum_{n=N}^{N+k} -t_n \left( \left( \frac{x}{R} \right)^n - \left( \frac{x}{R} \right)^{n+1} \right) \right] + t_{N-1} \left( \frac{x}{R} \right)^n - t_{N+k} \left( \frac{x}{R} \right)^{n+k+1}$$
Protože $x \in [0, R]$, tak $\left( \frac{x}{R} \right)^n \in [0,1]$. Dále platí $\left( \frac{x}{R} \right)^{n} - \left( \frac{x}{R} \right)^{n+1} \geq 0$.
\begin{eqnarray*}
\left|\sum_{n=N}^{N+k} a_n x^n \right| & \leq & \sum_{n=N}^{N+k} |t_n| \left( \left( \frac{x}{R} \right)^{n} - \left( \frac{x}{R} \right)^{n+1} \right) + |t_{N-1}| + |t_{N+k}| \\
& \leq & \varepsilon \sum_{n=N}^{N+k} |t_n| \left( \left( \frac{x}{R} \right)^{n} - \left( \frac{x}{R} \right)^{n+1} \right) + 2 \varepsilon \\
& = & \varepsilon \left( \left( \frac{x}{R} \right)^{N} - \left( \frac{x}{R} \right)^{N+k+1} \right) + 2 \varepsilon \\
& \leq & 3 \varepsilon 
\end{eqnarray*}
Z \emph{BC} podmínky pro stejnoměrnou konvergenci řady dostaneme $\sum_{n=0}^{\infty} a_n x^n \con$ na $[0, R]$
Z MO věty dostaneme
$$\lim_{n \rightarrow \infty} \lim_{r \rightarrow R_-} \sum_{n=0}^{N} a_n R^n = \lim_{r \rightarrow R_-} \lim_{n \rightarrow \infty} \sum_{n=0}^{N} a_n R^n$$
$$\lim_{n \rightarrow \infty} \lim_{r \rightarrow R_-} \sum_{n=0}^{N} a_n R^n = \lim_{n \rightarrow \infty} \sum_{n=0}^{N} a_n R^n = \sum_{n=0}^{\infty} a_n R^n$$
$$\lim_{r \rightarrow R_-} \lim_{n \rightarrow \infty} \sum_{n=0}^{N} a_n R^n = \lim_{r \rightarrow R_-} \sum_{n=0}^{\infty} a_n r^n$$
\end{proof}

\begin{priklad}
Sečtěte $\sum_{n=1}^{\infty} \frac{(-1)^{n-1}}{3n-2}$
\end{priklad}

\begin{proof}[Řešení]
Nechť $f(x) = \sum_{n=1}^{\infty} \frac{(-1)^{n-1}}{3n-2} x^{3n-2}$. To je mocninná řada poloměrem konvergence $R=1$. Podle Laibnitze $f(1) = \sum_{n=1}^{\infty} \frac{(-1)^{n-1}}{3n-2}$ konverguje. 

Tedy podle Abelovy věty $\sum_{n=1}^{\infty} \frac{(-1)^{n-1}}{3n-2} = \lim_{x \rightarrow 1_-} f(x)$

Dle věty o derivaci mocninné řady máme $\forall x \in (-1, 1)$
$$f'(x) = \sum_{n=1}^{\infty} \frac{(-1)^{n-1}}{3n-2} (3n-2) x^{3n-3} = \sum_{n=1}^{\infty} \left( -x^3 \right)^{n-1} = \frac{1}{1+x^3}$$
$$f(x) = \int \frac{1}{1+x^3} dx = \ldots = \frac{1}{3} \ln(x+1) - \frac{1}{6} \ln(x^2-x+1) + \frac{1}{\sqrt{3}} \arctan \left( \frac{2x-1}{\sqrt{3}} \right) + C$$
$$0 = f(0) = \frac{1}{3} 0 - \frac{1}{6} 0 + \frac{1}{\sqrt{3}} \arctan \left( - \frac{1}{\sqrt{3}} \right) + C \Rightarrow C = \frac{1}{\sqrt{3}} \arctan \left( \frac{1}{\sqrt{3}} \right)$$
\begin{eqnarray*}
\sum_{n=1}^{\infty} \frac{(-1)^{n-1}}{3n-2} & = & \lim_{x \rightarrow 1_-} \left( \frac{1}{3} \ln(x+1) - \frac{1}{6} \ln(x^2-x+1) + \frac{1}{\sqrt{3}} \arctan \left( \frac{2x-1}{\sqrt{3}} \right) \frac{1}{\sqrt{3}} \arctan \left( \frac{1}{\sqrt{3}} \right) \right) \\
& = & \frac{1}{3} \ln(2) + \frac{2}{\sqrt{3}} \arctan \left( \frac{1}{\sqrt{3}} \right) 
\end{eqnarray*}
\end{proof}
