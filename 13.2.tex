\begin{definice}
Nehcť $z_0 \in \mathbb{R}$ a $a_n \in \mathbb{C}$ pro $n \in \mathbb{N}_0$. Řadu funkcí $\sum_{n=0}^{\infty} a_n (z-z_0)^n$ pro $z \in \mathbb{C}$ nazýváme mocninnou řadou s koeficienty $a_n$ o středu $z_0$.
\end{definice}

\begin{vetat}[o rozvoji do Taylorovy řady]
\label{o rozvoji do Taylorovy řady}
Nechť $f$ je holomorfní na kruhu $B(z_0, R)$. Pak existuje právě jedna mocninná řada s poloměrem konvergence alespoň $R$, že na $B(z_0, R)$ platí
$$f(z) = \sum_{n=0}^\infty a_n (z-z_0)^n$$
Navíc platí $a_n = \frac{f^{(n)}(z_0)}{n!}$ pro všechna $n \in \mathbb{N}_0$.
\end{vetat}

Jako u reálných mocninných řad lze na kruhu konvergence prohazovat $\sum$ a derivaci a důkaz je podobný.

\begin{dusledek}
Je-li $f$ holomorfní na $G$, pak na $G$ existují derivace všech řádů $f^{(k)}$ pro $k \in \mathbb{N}$.
\end{dusledek}

\begin{proof}
\underline{jednoznačnost}: 
\begin{eqnarray*}
f(z) & = & \sum_{n=1}^\infty a_n (z-z_0)^n \\
f^\prime(z) & = & \sum_{n=1}^\infty a_n n (z-z_0)^{n-1}
\end{eqnarray*}
$f(z)$ k-krát zderivujeme. Připomeňmě, že pro mocninnou řadu platí $(\sum)^\prime = (\sum ^\prime)$ ze stejnoměrné konvergence.
\begin{eqnarray*}
f^{(k)}(z) & = & \sum_{n=k}^\infty a_n n (n-1) \ldots (n-k+1)(z-z_0)^{n-k}
\end{eqnarray*}
dosadíme $z=z_0$ : $f^{(k)}(z_0) = a_k k!$ ($n=k$)

\underline{existence} :
$f$ je holomorfní na $B(z_0, R)$, nechť $0 < r < R$ je pevné a $\varphi (t) = z_0 + re^{it}$. Nechť $z \in B(z_0, r)$. Z Cauchyova vzorce : $f(z) = \frac{1}{2 \pi i} \int_\varphi \frac{f(s)}{s-z} ds$.
$$\frac{1}{s-z} = \frac{1}{(s-z_0)-(z-z_0)} = \frac{1}{s-z} \frac{1}{1 - \frac{z-z_0}{s-z_0}} \overset{|z-z_0|<|s-z_0|=r}{=} \frac{1}{s-z_0} \sum_{n=0}^\infty \left( \frac{z-z_0}{s-z_0} \right)^n$$
$$\frac{1}{2 \pi i} \int_\varphi \sum_{n=0}^\infty f(s) \frac{(z-z_0)^n}{(s-z_0)^{n+1}} ds \overset{\sum \con \textrm{ na } \langle \varphi \rangle = S(z_0, r)}{=} \frac{1}{2 \pi i} \sum_{n=0}^\infty (z-z_0)^n \int_\varphi \frac{f(s)}{(s-z_0)^{n+1}} ds$$
$$a_n = \frac{1}{2 \pi i} \int_\varphi \frac{f(s)}{(s-z_0)^{n+1}}$$
$$\sup_{s \in \langle \varphi \rangle} \left| f(s) \frac{(z-z_0)^n}{(s-z_0)^{n+1}} \right| \leq \max_{\langle \varphi \rangle} \frac{1}{r} q^n \quad \textrm{pro } q = \frac{|z-z_0|}{|s-z_0|} < 1$$
$$\sum \max f \frac{1}{r} q^n \textrm{ konverguje } \overset{Weirstrass}{\Rightarrow} \sum \con$$
\end{proof}


\begin{definice}
Množina $G \subset \mathbb{C}$ se nazývá \emph{souvislá}, pokud pro všechny $A, B \subset G$ otevřené v $G$ platí
$$A \cap B = G \textrm{ a } A \cup B = \emptyset \Rightarrow A = \emptyset \textrm{ nebo } B = \emptyset$$
Množina $G \subset \mathbb{C}$ se nazývá \emph{oblast}, pokud je otevřená a souvislá.
\end{definice}

\begin{vetal}[o jednoznačnosti holomorfní funkce]
Nechť $G \subset \mathbb{C}$ je oblast a $f, g$ jsou holomorfní na $G$. Předpokládejme, že množina
$$M = \left\{ z \in G : f(z)=g(z) \right\} $$
má hromadný bod v $G$, neboli existují $z_n \in M$ a $z_0 \in G$ takové, že $z_n \stackrel{n \rightarrow \infty}{\rightarrow} z_0$. Pak $f=g$ na $G$.
\end{vetal}

\begin{dusledek}
$\sin^2 (z) + \cos^2 (z) = 1$ platí $\forall z \in \mathbb{C}$, neboť platí na $\mathbb{R}$ - reálná osa $\to$ úsečka
\end{dusledek}

\begin{proof}
Bez újmy na obecnosti předpokládejme $g=0$ a $z_0=0$ (jinak posuneme oblast).

Nechť $f(z) = \sum_{n=1}^\infty a_n z^n$ (lze dle věty \ref{o rozvoji do Taylorovy řady})

Pokud $a_n = 0$, $\forall n \Rightarrow \surd$

Nechť $\exists n_0 \textrm{ : } a_n \neq 0$, první takové. 
$$f(z) = a_{n_0} z^{n_0} + a_{n_0 + 1} z^{n_0 + 1} + \ldots = z^{n_0} \left( a_{n_0} + \underbrace{a_{n_0 + 1} z + \ldots}_{\lim_{z \to 0} = 0 \textrm{ ze spojitosti}} \right)$$

Tedy k $\varepsilon = \frac{|a_{n_0}|}{2} \ \exists \delta \ \forall z \textrm{ : } |z| < \delta (a_{n_0 + 1} z + \ldots) < \frac{|a_{n_0}|}{2}$, tedy $|f(z)| > |z|^{n_0} \frac{|a_{n_0}|}{2}$ spec. $f \neq 0$ na $B(0, \delta) \backslash \{ 0 \}$ spor s $0$ je hromadný bod $\{ f = g = 0 \}$ (1. člen $a_{n_0}$ je nenulový)
\end{proof}


\begin{definice}
Řekneme, že funkce $f$ má v bodě $z_0$ pól násobnosti nejvýše $k \in \mathbb{N}$, je-li funkce 
\begin{equation}
F(z) = \left\{ \begin{array}{ll}
 (z-z_0)^{k+1}f(z) & \textrm{pro $z \neq z_0$} \nonumber\\
 0 & \textrm{pro $z=z_0$}
  \end{array} \right.
\end{equation}

holomorfní na nějakém okolí bodu $z_0$. Řekneme, že má pól násobnosti $k$, je-li $k \in \mathbb{N}$ nejmenší s touto vlastností.
\end{definice}

Například funkce $f(z) = 1 / z^k$ má v bodě 0 pól násobnosti k.

\begin{definice}
Nechť $M \subset G \subset \mathbb{C}$ je konečná množina. Řekneme, že funkce $f : G \backslash M \rightarrow \mathbb{C}$ je \emph{meromorfní} v $G$, pokud je $f$ holomorfní na $G \backslash M$ a v bodech $M$ má $f$ póly (konečné násobnosti).
\end{definice}

\begin{center}
\includegraphics[scale=0.7]{obrazky.2}
\end{center}


\begin{vetat}[o rozovji do Laurentovy řady]
Nehcť $f$ je holomorfní na mezikruží $B(z_0, R) \backslash \overline{B(z_0, r)}$, $0 < r < R$. Pak existují jednoznačně určená čísla $a_k \in \mathbb{C}$, $k \in \mathbb{Z}$, že platí 
$$f(z) = \sum_{k= - \infty}^\infty a_k (z-z_0)^k \textrm{ pro všechna } z \in B(z_0, R) \backslash \overline{B(z_0, r)}$$
\end{vetat}

\begin{proof}
Platí Cauchyho vzorec pro mezikruží:
$$f(z) = \frac{1}{2 \pi i} \int_{\varphi_R} \frac{f(s)}{s-z} ds - \frac{1}{2 \pi i} \int_{\varphi_r} \frac{f(s)}{s-z} ds$$

\begin{figure}[!h] \begin{center}
\includegraphics{obrazky.3}
\end{center} \end{figure}

\begin{equation*}
F(z) = \left\{ \begin{array}{ll}
 \frac{f(s)-f(z)}{s-z} & \textrm{pro $s \neq z$, $F$ holomorfní na $B_R \backslash B_r$ a spojitá v $z$} \\
 f^\prime(z) & \textrm{pro $z=z_0$, $\Rightarrow \int_{\varphi_1} F(z) = 0 = \int_{\varphi_2} F(z) dz$}
  \end{array} \right.
\end{equation*}

sečtením $\int_{\varphi_R} F - \int_{\varphi_r} F = 0$
$$\int_{\varphi_R} \frac{f(s)}{s-z} ds - f(z) \underbrace{\int_{\varphi_R} \frac{1}{s-z} ds}_{2 \pi i} - \int_{\varphi_r} \frac{f(s)}{s-z} ds + f(z) \underbrace{\int_{\varphi_r} \frac{1}{s-z} ds}_{0} \Rightarrow \textrm{Cauchyův vzorec pro mezikruží}$$
$$2 \pi i f(z) = \int_{\varphi_R} \frac{f(s)}{s-z} ds - \int_{\varphi_r} \frac{f(s)}{s-z} ds$$
$$\frac{1}{s-z} = \frac{1}{s-z_0} \sum_{n=0}^\infty \left( \frac{z-z_0}{s-z_0} \right)^n \to \int_{\varphi_R} \frac{f(s)}{s-z} = \sum_{k=0}^\infty a_k (z-z_0)^k \quad \textrm{viz důkaz věty \ref{o rozvoji do Taylorovy řady}}$$

na $\varphi_r$ : $|z-z_0| > |z-s|$ $\forall s \in \langle \varphi_r \rangle$
\begin{figure}[!h] \begin{center}
\includegraphics{obrazky.7}
\end{center} \end{figure}
$$\frac{1}{s-z} = \frac{1}{(s-z_0)-(z-z_0)} = \frac{-1}{z-z_0} \frac{1}{1 - \frac{s-z_0}{z-z_0}} = - \frac{1}{z-z_0} \sum_{n=0}^\infty \left( \frac{s-z_0}{z-z_0} \right)^n = \sum_{k=-1}^{-\infty} (z-z_0)^k \frac{1}{(s-z_0)^{k+1}}$$
Nyní
$$- \int_{\varphi_r} \frac{f(s)}{s-z} ds = \int_{\varphi_r} \sum_{k=-1}^{-\infty} (z-z_0)^k \frac{1}{(s-z_0)^{k+1}} f(s) ds \overset{\circledast}{=} \sum_{k=-1}^{-\infty} (z-z_0)^{k} \int_{\varphi_r} \frac{f(s)}{(s-z_0)^{k+1}} ds$$

$\circledast$ analogicky jako předtím $\sum \con$ z Weirstrasse
\end{proof}
